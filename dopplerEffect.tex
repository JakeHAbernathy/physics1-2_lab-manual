\section{Doppler Effect}

Supplies:
\begin{itemize}
\item wheel thingy
\item a few meter sticks
\item stop watch with lap function
\item pencil/paper
\end{itemize}

Instructions:
\begin{itemize}
\item Form into two group of 8-10 students.
\item Choose reasonable values for:
\begin{itemize}
\item wave propagation velocity, $v$
\item wavelength, $\lambda$
\item velocity of the observer, $v_s$
\end{itemize}
\item From these values calculate the expected unshifted period, $T$, and the observed (shifted) period $T^\prime$.
\item Have 5-6 students walk in a single file line with velocity $v$, separated by a distance $\lambda$, holding out their hands for a high-five.
\item Choose an observer from your group. The line of 5-6 students should then high-five the observer as they walk by. Do this twice - once with a stationary observer and again with the observer moving towards the line with velocity $v_s$.
\item While the experiment is running, have another student use a stopwatch to record the time interval between high-fives. Record several time intervals for each experiment and average them.
\item Compare your calculated $T$ and $T^\prime$ with your measured $T$ and $T^\prime$.
\end{itemize}

\pagebreak \clearpage
